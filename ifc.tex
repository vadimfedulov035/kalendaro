\documentclass[a4paper,class=article,tikz,border=0pt]{standalone}

\usepackage[pass,a4paper]{geometry}
\usepackage[utf8]{inputenc}
\usepackage{tikz}
\usepackage{anyfontsize}
\usepackage{arrayjob}
\usepackage{calculator}
\usepackage{ifthen}
\usepackage{datenumber}
\usepackage{advdate}
\usepackage{minibox}
\usepackage{xcolor}
\usepackage{xparse}
\usepackage{pgfmath}

\usepackage{fontspec}
\setmainfont{DejaVu Sans}

\definecolor{Esperanto}{rgb}{0,0.8682,0.1318}

\newcommand\plena{0}  % Generate annual calendar (1) or month (0)
\newcommand\jaro{2024}  % What year to generate (nothing for eternal calendar)
\newcommand\monato{6}  % What month to generate (works only with month calendar)
\newcommand\tago{4}  % What day to highlight (nothing for default look)

\pgfkeyssetvalue{/monatoj/1}{Januaro}
\pgfkeyssetvalue{/monatoj/2}{Februaro}
\pgfkeyssetvalue{/monatoj/3}{Marto}
\pgfkeyssetvalue{/monatoj/4}{Aprilo}
\pgfkeyssetvalue{/monatoj/5}{Majo}
\pgfkeyssetvalue{/monatoj/6}{Junio}
\pgfkeyssetvalue{/monatoj/7}{Sunio}
\pgfkeyssetvalue{/monatoj/8}{Julio}
\pgfkeyssetvalue{/monatoj/9}{Aŭgusto}
\pgfkeyssetvalue{/monatoj/10}{Septembro}
\pgfkeyssetvalue{/monatoj/11}{Oktobro}
\pgfkeyssetvalue{/monatoj/12}{Novembro}
\pgfkeyssetvalue{/monatoj/13}{Decembro}

\newarray\tagsemajno
\readarray{tagsemajno}{dimanĉo&lundo&mardo&merkredo&ĵaŭdo&vendredo&sabato}

\newarray\astrosemajno
\readarray{astrosemajno}{Suno ☉&Luno ☾&Marso ♂&Merkuro ☿&Jupitero ♃&Venuso ♀&Saturno ♄}

\input{eo.ldf}

\NewDocumentCommand{\framecolorbox}{oommm}
 {% #1 = width (optional)
  % #2 = inner alignment (optional)
  % #3 = frame color
  % #4 = background color
  % #5 = text
  \IfValueTF{#1}
   {\IfValueTF{#2}
	{\fcolorbox{#3}{#4}{\makebox[#1][#2]{#5}}}
	{\fcolorbox{#3}{#4}{\makebox[#1]{#5}}}%
   }
   {\fcolorbox{#3}{#4}{#5}}%
 }

\newcommand\xsize[2]{\fontsize{#1}{
	\MULTIPLY{#1}{1.2}{\distanco}
	\distanco
}\selectfont#2}
\newcommand\kromtago[1]{
	\node at (277/7*6 mm + 10mm + 277mm / 2 / 7, 204mm - 135/4mm) {\Large #1};
	\node [anchor=west] at (13mm + 277/7*6 mm, 194mm - 135/4mm) {\Large 29};
	\draw[gray] (277/7*6 mm + 10mm, 200mm - 135/4 mm) -- (287mm, 200mm - 135/4 mm);

	% Is calendar annual?
	\ifthenelse{\equal{\jaro}{}}{}{
		\node[gray,anchor=west] at (277/7*6 mm + 13mm, 170mm - 135/4mm) {\thedateday-a de \datemonthname};
		\nextdate
	}
}

% LEAP YEAR
\MODULO{\jaro}{4}{\xmodUNU}
\MODULO{\jaro}{100}{\xmodDU}
\MODULO{\jaro}{400}{\xmodTRI}
\ifthenelse{\equal{\xmodUNU}{0}\AND\NOT\equal{\xmodDU}{0}\OR\equal{\xmodTRI}{0}}{
	\newcommand\superjaro{1}
}{
	\newcommand\superjaro{0}
}

\begin{document}

	% Set date for annual calendar
	\ifthenelse{\equal{\jaro}{}}{}{
		\setdate{\jaro}{1}{1}
	}
	
	% Design
	\newcommand\printu[1]{
		\begin{tikzpicture}
			% Background
			\draw[transparent] (0,0) rectangle (297mm,210mm);

			% Title with month and year
			\node [anchor=north west,align=left] at (10mm,200mm) {
				\xsize{80}{\pgfkeysvalueof{/monatoj/#1} \jaro}
				\hspace{-.5em}
				\ifthenelse{\equal{\plena}{1}\AND\equal{\pgfkeysvalueof{/monatoj/#1}}{Januaro}\OR\equal{\jaro}{}}{
					\minibox{
						\\[-.25em]
						\xsize{20}{Internacia} \\[.4em]
						\xsize{20}{Fiksita} \\[.4em]
						\xsize{20}{Kalendaro}
					}
				}{}
			};

			% LINES

			% Horizontal lines
			\foreach \y in {0,...,4} {
				 \draw[gray] (10mm, 10mm + 135mm - 135/4*\y mm) -- (287mm, 10mm + 135mm - 135/4*\y mm);
			}
			% Vertical lines
			\foreach \x in {0,...,7} {
				\draw[gray] (10mm + 277/7*\x mm, 145mm) -- (10mm + 277/7*\x mm, 10mm);
			}

			% Highlight day
			\ifthenelse{\equal{\tago}{}}{}{
				\DIVIDE{\tago}{7}{\semajno}
				\MODULO{\tago}{7}{\semajntago}
				\ifthenelse{\equal{\semajntago}{0}}{
					\renewcommand\semajntago{7}
				}{}

				% Horizontal highlight lines 
				\pgfmathsetmacro\supralinio{5 - ceil(\semajno)}
				\pgfmathsetmacro\malsupralinio{\supralinio - 1}

				% Vertical hightlight lines
				\newcommand\dekstralinio{\semajntago}
				\pgfmathsetmacro\maldekstralinio{\dekstralinio - 1}

				% Handle leap and new year day (29-th) 
				\ifthenelse{\not\equal{\tago}{29}}{
					% Horizontal highlight lines
					\draw[line width=0.75mm][Esperanto] (10mm + 277/7*\maldekstralinio mm - 1, 10mm + 135/4*\supralinio mm) -- (10mm + 277/7*\dekstralinio mm + 1, 10mm + 135/4*\supralinio mm);
					\draw[line width=0.75mm][Esperanto] (10mm + 277/7*\maldekstralinio mm - 1, 10mm + 135/4*\malsupralinio mm) -- (10mm + 277/7*\dekstralinio mm + 1, 10mm + 135/4*\malsupralinio mm);
					% Vertical highlight lines
					\draw[line width=0.75mm][Esperanto] (10mm + 277/7*\maldekstralinio mm, 10mm + 135/4*\malsupralinio mm - 1) -- (10mm + 277/7*\maldekstralinio mm, 10mm + 135/4*\supralinio mm + 1);
					\draw[line width=0.75mm][Esperanto] (10mm + 277/7*\dekstralinio mm, 10mm + 135/4*\malsupralinio mm - 1) -- (10mm + 277/7*\dekstralinio mm, 10mm + 135/4*\supralinio mm + 1);
				}{
					\renewcommand\supralinio{5}
					\renewcommand\malsupralinio{4}
					\renewcommand\maldekstralinio{6}
					\renewcommand\dekstralinio{7}
					% Horizontal highlight lines (exception)
					\draw[line width=0.75mm][Esperanto] (10mm + 277/7*\maldekstralinio mm - 1, 27mm + 135/4*\supralinio mm) -- (10mm + 277/7*\dekstralinio mm + 1, 27mm + 135/4*\supralinio mm);
					\draw[line width=0.75mm][Esperanto] (10mm + 277/7*\maldekstralinio mm - 1, 27mm + 135/4*\malsupralinio mm) -- (10mm + 277/7*\dekstralinio mm + 1, 27mm + 135/4*\malsupralinio mm);
					% Vertical highlight lines (exception)
					\draw[line width=0.75mm][Esperanto] (10mm + 277/7*\maldekstralinio mm, 27mm + 135/4*\malsupralinio mm - 1) -- (10mm + 277/7*\maldekstralinio mm, 27mm + 135/4*\supralinio mm + 1);
					\draw[line width=0.75mm][Esperanto] (10mm + 277/7*\dekstralinio mm, 27mm + 135/4*\malsupralinio mm - 1) -- (10mm + 277/7*\dekstralinio mm, 27mm + 135/4*\supralinio mm + 1);
				}
			}

			% INFORMATION
			% Day of the week
			\foreach \x in {1,...,7} {
				\node at (10mm + 277/7*\x mm - 277mm / 2 / 7, 156mm) {\Large\astrosemajno(\x)};
			}
			
			\foreach \x in {1,...,7} {
				\node at (10mm + 277/7*\x mm - 277mm / 2 / 7, 150mm) {\Large\tagsemajno(\x)};
			}

			% SUPERDAY SHIFT IN GREGORIAN DAYS
			% If specified calendar is annual there is no shift
			\ifthenelse{\equal{\plena}{1}}{}{
				% Otherwise calculate the shift
				\ifthenelse{\equal{\superjaro}{1}\AND{#1>6}}{
				% leap year case after June (month * 28 + 1)
					\pgfmathsetmacro\aldontagoj{(#1-1)*28+1}
				}{
				% ordinary year (month * 28)
					\pgfmathsetmacro\aldontagoj{(#1-1)*28}
				}
				% Add shift day number to date counter
				\addtocounter{datenumber}{\aldontagoj}%
				\setdatebynumber{\thedatenumber}%
			}

			% PARALLEL IFC AND GREGORIAN DAYS
			\foreach \n in {0,...,27} {
				% IFC day
				\ADD{\n}{1}{\N}
				\INTEGERDIVISION{\n}{7}{\line}{\xmod}
				\node [anchor=west] at (15mm + 277/7*\xmod mm, 140mm - 135/4*\line mm) {\large\N};
				% Gregorian day
				\ifthenelse{\equal{\jaro}{}}{}{
					\node[gray,anchor=west] at (15mm + 277/7*\xmod mm, 115mm - 135/4*\line mm) {\thedateday-a de \datemonthname};
					\nextdate
				}
			}

			% ALL IFC EXTRA DAYS

			% LEAP DAY
			\ifthenelse{\equal{\superjaro}{1}\AND{\equal}{#1}{6}}{
			% Only if year is leap
				\kromtago{supertago}
			}{}
			% Always if annual calendar
			\ifthenelse{\equal{\jaro}{}\AND{\equal}{#1}{6}}{
				\kromtago{supertago}
			}{}

			% YEAR DAY
			\ifthenelse{\equal{#1}{13}}{
				\kromtago{jartago}
			}{}
		\end{tikzpicture}
		}

	\ifthenelse{\equal{\plena}{1}}{
		\foreach \m in {1,...,13} {
			\printu{\m}
		}
	}{
		\printu{\monato}
	}
\end{document}
